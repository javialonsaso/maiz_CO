\documentclass[12pt, spanish]{article}
\usepackage[left = 2.5cm, right = 2.5cm, bottom = 2.5cm, top = 2.5cm]{geometry}
\usepackage[spanish]{babel}
\usepackage{graphicx}
\usepackage[colorlinks=true, linkcolor=blue]{hyperref}
\usepackage[utf8]{inputenc}
\usepackage[svgnames]{xcolor}
\usepackage{appendix}
\newcommand\tab[1][1cm]{\hspace*{#1}}
\usepackage{amsmath}
\usepackage{subcaption}
\renewcommand{\tablename}{Tabla}
\renewcommand{\figurename}{Figura}

%comandos para poner la bibliografía con estilo de cita harvard
\renewcommand{\refname}{Referencias} %para que ponga referencias en el apartado de referencias y no el término en inglés
\renewcommand{\citeform[1]}{\textcolor{blue}{#1}}% las citas en el texto salen de color azul

%líneas exclusivas de la implementación del estilo harvard
%usepackage[sort&compress]{natbib}
%\bibliographystyle{abbrvnat}
%\setcitestyle{authoryear,open={(},close={)}}

\renewcommand{\contentsname}{Índice}
\usepackage{listings}
\usepackage{afterpage}
\pagestyle{headings}%este estilo de página permite imprimir el nombre del capítulo o sección que se está procesando en la cabecera junto con el número de página en el lado opuesto, permaneciendo el pie de página vacio.
\usepackage{color}
\setlength{\parskip}{1em} %con esta línea dejamos una línea en blanco entre párrafos



\begin{document}

\pagenumbering{roman}
\tableofcontents 
\newpage
\pagenumbering{arabic}
\setcounter{page}{1}

\section{Punto de partida del proyecto}

Nuestros datasets:

1. producción(formato csv)
https://www.datos.gov.co/Agricultura-y-Desarrollo-Rural/Cadena-Productiva-Ma-z-Area-Producci-n-Y-Rendimien/d968-yfb5
2. imágenes de temperatura (formato tiff, imagen georeferenciada);
https://developers.google.com/earth-engine/datasets/catalog/MODIS_006_MOD11A1
3.polígonos de veredas que contienen municipios (formato shp).
https://geoportal.dane.gov.co/servicios/descarga-y-metadatos/descarga-nivel-de-referencia-de-veredas/

\section{Descripción general del proyecto}

\subsection{Objetivos}

Comparar la producción de Zea mays (maíz) en función de la temperatura entre los diferentes municipios de Colombia (serán las áreas de producción). Para comparar la producción usaremos el cociente de toneladas producidas por hectárea (rendimiento). También compararemos la eficiencia de la tierra (proporción del terreno sembrado que genera producto)

Todo ello para determinar la zona(municipio) y temperatura de mayor producción de zea mays en Colombia.

\subsection{Resultados esperados}

Conforme aumente la temperatura se espera una producción mayor , con una temperatura asociada a una máxima producción de unos 25 grados centígrados. A temperaturas inferiores el rendimiento es menor llegando a una producción nula para temperaturas muy bajas.

\subsection{Requisitos}

Requisitos: aquí hablamos sobre los datos que tenemos (es nuestro punto de partida, de arriba, lo que hemos puesto de los formatos y a qué queremos llegar a partir de ellos).

1. producción: cadena productiva de maíz. En este csv podemos encontrar el rendimiento de la producción de Zea mays (maíz) en unidades de toneladas/hectárea por cada municipio de Colombia. Los datos se dividen por semestres.
2. imágenes de temperatura: temperatura por vereda (más pequeño que un municipio, un municipio se divide en veredas) para cada día. Para cada día tenemos una temperatura máxima y otra mínima. (importante: la necesitamos semestralmente porque los datos de la cosecha están por semestres, para ello hacemos una media de cada día, y después con las medias de todos los días, hacemos la media de cada semestre, obteniendo así 2 temperaturas por año)
3. polígonos de veredas que contienen municipios: tenemos las temperaturas por vereda, y la producción por municipio. Entonces debemos agrupar las temperaturas de las veredas a cada respectivo municipio para ya tener las temperaturas semestrales por municipio.

\subsection{Requerimientos técnicos}

(describir aquí las condiciones de cultivo del maiz por encima, entre otras cosas los 94 días que dura el ciclo de cultivo del maíz).

Para poder generar las fuentes de datos necesarias para el proyecto es necesario establecer el cultivo del maíz en las diferentes regiones de Colombia, para ello se deberá disponer lo siguiente:

- Contratación de personal cualificado para el cultivo (agricultores).

- Un calendario de siembra y cosecha para el cultivo adaptado a cada región de Colombia e información sobre la temperatura adecuada para el desarrollo del maíz.

Atendiendo a la bibliografía encontrada se ha realizado una agrupación de los diferentes municipios de Colombia por zonas: 
ZAE1, ZAE2, ZAE3, ZAE4, ZAE5  y ZAE6, esto con el objetivo de reducir la variabilidad en las condiciones de siembra y cosecha. Planteándose el siguiente calendario para la siembra:

y el siguiente calendario para la cosecha: 

- Análisis químico del suelo en cada una de las zonas geográficas para determinar las necesidades nutricionales del mismo y por tanto los complementos necesarios para el desarrollo del cultivo del maíz. Estos complementos se tendrán en cuenta en la elección del abono apropiado para el campo de cultivo.

- Sistema de riego optimizado para aquellas regiones donde las precipitaciones son escasas.

- Contratación del servicio de estaciones meteorológicas con el gobierno de Colombia para disponer de los datos de temperatura de las regiones durante el transcurso del cultivo. 

- Contratación del servicio de imágenes de satélite con el gobierno de Colombia con el objeto de delimitar el área de las regiones a estudiar. 

\subsection{Descripción del problema}

\subsubsection{Interés}

- Importancia y usos del cultivo del maiz.

- Qué es Zea mays, en qué se diferencia del maiz original, por qué es tan importante en Colombia.

- Qué ventajas tiene Zea mays respecto al maíz original. 

- Colombia, condiciones ideales para el cultivo de Zea mays en general, teniendo en cuenta esto elegir la mejor zona si se dispone de recursos limitados, por ejemplo solo se puede abarcar terreno en uno de los municipios, para así asegurar la producción más efectiva a nivel geográfico, decidiendo esto en base a la condición temperatura de la zona, pues parece ser que en Colombia es un aspecto muy variable entre las diferentes regiones.

El problema a enfocar es encontrar el municipio de Colombia óptimo para la producción de Zea Mays, además de la temperatura asociada para dicha producción óptima. En definitiva, optimizar la producción de maíz en Colombia eligiendo la mejor región de producción.
        
(TEMPORAL) 
El cambio climático es un problema global que afecta a varios sectores de la industria. En específico, la alimentaria es una de las más relevantes y la que es más afectada por cambios en la climatología. Comprender el estado actual de la producción agrícola teniendo en cuenta. Javi? qué querías decir aquí?

\subsubsection{Cobertura (geográfica, temporal)}

La cobertura temporal abarca el período 2015-2018, dividiendo los años por   semestres. Concretamente para 2018 las medidas finalizan con el primer semestre.

En cuanto a la cobertura geográfica se centra en Colombia y en los municipios productores del cultivo elegido.

La limitación estaría centrada en el período estudiado para la producción, se ofrecen esos años y nada más.

\subsubsection{Objetivos}

Parafrasear los objetivos del principio.

\subsubsection{Descripción de las fuentes de los datos}

Los datos han sido extraídos de las siguientes bases de datos en abierto: 

- Base de datos en abierto del gobierno de Colombia (https://www.datos.gov.co/). Concretamente los datos de producción han sido oobtenidos de la división de agricultura y desarrollo rural.

- The Land Processes Distributed Active Center (LP DACC), que es una base de datos en abierto englobada dentro del sistema de observación de la NASA (NASA Earth Observing System Data and Information System, EOSDIS) en la división USGS Earth Resources Observation and Science (EROS) (https://lpdaac.usgs.gov/).

- Geoportal del Departamento Aministrativo Nacional de Estadística de Colombia (DANE) (https://geoportal.dane.gov.co/)

los formatos en los que se han recogido los datos han sido los siguientes:

- .csv para los datos de producción.

- ...

\subsection{Data Management plan}

1. Data summary

- Propósito de la recogida / generación de datos y su relación con los objetivos del proyecto.

Encontrar una relación entre temperatura y rendimiento de la producción de maíz que permita decidir cuál es el mejor municipio de Colombia para alcanzar una mayor producción de maíz.

- ¿Qué tipos y formatos de datos generará o recogerá  el proyecto?

- ¿Usaremos datos ya existentes? ¿cómo?

Los datos a usar no van a ser datos ya existentes, pues comenzaremos el proceso desde 0, desde el sembrado, pasando por la instalación de las estaciones metereológicas que nos permitirán controlar la temperatura hasta la toma de imágenes de satélite de los municipios de Colombia estudiados.

- ¿Cuál es el origen de los datos?

Los datos no tienen un único origen, provienen de tres fuentes diferentes: estaciones meteorológicas, análisis de la producción de maiz e imágenes de satélite de las regiones estudiadas. 

poner por aquí que cada una de estas ramas de datos ha sido depositada en el repositorio correspondiente: gobierno de colombia, dane , etc.

- ¿Cuál es el tamaño esperado de los datos?
no entiendo a qué se refiere con esto.
- ¿Para quién podrían ser útiles estos datos?

Estos datos podrían ser útiles para pequeños agricultures interesados en comenzar a producir maíz, pues la relación temperatura-producción, les permitiría saber si una zona es óptima o no. Reduciendo con mucho el riesgo inicial de producción, no se arriesgan a sembrar en una zona no adaptada climatológicamente para la producción de maíz. Así en el plazo de un año podrían tener ganancias fijas que les permitiesen ampliar zona de cultivo en caso de ser necesario.

También sería posible saber para una nueva zona, no tiene por qué ser uno de los municipios dados, conociendo su temperatura media si es adecuada o no para la producción.

2. Fair Data

Hacer los datos accesibles, incluyendo metadatos

- ¿Se pueden encontrar los datos producidos y / o usados en el proyecto con los metadados, identificables y localizables por medio de un mecanismo  de identificación estándar? (por ejemplo, identificadores persistentes y únicos como Digital Object Identifiers (DOIs)?

Los datos pueden ser encontrados en el repositorio de github(), junto con este informe en el que se describe su obtención, uso y análisis. 

- ¿Qué convenciones de nombres se han seguido?

- ¿Se proporcionan palabras clave para optimizar la posibilidades de reutilización?

mirar metadados sil

- ¿Qué metadados se han creado? En caso de que no haya metadatos estándar en la disciplina, indicar el tipo de metadado y cómo se ha creado.

\subsection{Curación de Datos y ETL}

Las tres fuentes de datos tuvieron que pasar por un proceso de curación y posteriormente por un proceso ETL para obtener una base de datos limpia y lista para realizar cualquier tipo de análisis sobre ella.

\subsubsection{Preprocesamiento de imágenes}

Las imágenes georreferenciadas de temperatura superficial, obtenidas a partir del producto MOD11A1, del sensor MODIS abordo del satélite de la NASA Terra EOS AM-1, fueron las primeras en preprocesarse, ya que era necesario adecuar la temporalidad de estos datos (diaria) a la temporalidad de los datos de producción (semestral).

Cada imagen de temperatura superficial está compuesta por 12 sub-imágenes, conocidas como bandas, de las cuales se utilizaron dos:

1. LST_Day_1km
2. LST_Night_1km

Estas bandas contienen la información de la temperatura diaria de la superficie terrestre de día y de noche repartidas en una grilla de 1 km. Ambas bandas fueron utilizadas como temperatura superficial mínima y temperatura superficial máxima y con ellas calcular la temperatura superficial media por cada valor de pixel de la grilla.

Teniendo una imagen de temperatura superficial por día, se calculó el promedio por cada pixel de la grilla se manera semestral desde el año 2015 hasta el primer semestre del año 2018 (fecha hasta la cual se encuentran disponibles los datos de producción de maíz). Por tal motivo se redujo la cantidad de imágenes diarias a 7 imágenes correspondientes a la temperatura superficial media semestral de los años correspondientes para la cobertura espacial de Colombia.

\subsubsection{Preprocesamiento de veredas}

Colombia, hasta el nivel de veredas, se encuentra administrativamente dividida en 3 niveles:

- Nivel Superior: Departamentos.
- Nivel Intermedio: Municipios.
- Nivel Inferior: Veredas.

El nivel en el que se encuentran los datos de producción de maíz es el de municipios, sin embargo, el dataset de municipios georreferenciados no se encuentra disponible en los datos abiertos ofrecidos por el Gobierno Colombiano, por tal razón, se han elegido los datos correspondientes a las veredas, que se encuentran a un nivel inferior y pueden ser transformados a un dataset georreferenciado de municipios en un simple paso.

El dataset georreferenciado de veredas contiene asociado una base de datos, en donde cada vereda se encuentra dentro de un municipio. Al tener todas las veredas un municipio asociado, para datos geoespaciales se puede realizar la acción de 'disolver'. Disolver datos espaciales permite generar polígonos de mayor tamaño con respecto a una columna en común, por tal motivo, todas las verdeas que pertenezcan a un mismo municipio, se unirán y generarán un polígono más grande que corresponde a dicho municipio.


\subsection{Metadatos}

\subsection{Plan de Preservación}

\subsection{Análisis de los Datos}


































\bibliography{referencias_informe.bib}
\bibliographystyle{apalike}
\end{document}
